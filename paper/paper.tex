\documentclass[]{elsarticle} %review=doublespace preprint=single 5p=2 column
%%% Begin My package additions %%%%%%%%%%%%%%%%%%%
\usepackage[hyphens]{url}

  \journal{An awesome journal} % Sets Journal name


\usepackage{lineno} % add
\providecommand{\tightlist}{%
  \setlength{\itemsep}{0pt}\setlength{\parskip}{0pt}}

\usepackage{graphicx}
\usepackage{booktabs} % book-quality tables
%%%%%%%%%%%%%%%% end my additions to header

\usepackage[T1]{fontenc}
\usepackage{lmodern}
\usepackage{amssymb,amsmath}
\usepackage{ifxetex,ifluatex}
\usepackage{fixltx2e} % provides \textsubscript
% use upquote if available, for straight quotes in verbatim environments
\IfFileExists{upquote.sty}{\usepackage{upquote}}{}
\ifnum 0\ifxetex 1\fi\ifluatex 1\fi=0 % if pdftex
  \usepackage[utf8]{inputenc}
\else % if luatex or xelatex
  \usepackage{fontspec}
  \ifxetex
    \usepackage{xltxtra,xunicode}
  \fi
  \defaultfontfeatures{Mapping=tex-text,Scale=MatchLowercase}
  \newcommand{\euro}{€}
\fi
% use microtype if available
\IfFileExists{microtype.sty}{\usepackage{microtype}}{}
\bibliographystyle{elsarticle-harv}
\ifxetex
  \usepackage[setpagesize=false, % page size defined by xetex
              unicode=false, % unicode breaks when used with xetex
              xetex]{hyperref}
\else
  \usepackage[unicode=true]{hyperref}
\fi
\hypersetup{breaklinks=true,
            bookmarks=true,
            pdfauthor={},
            pdftitle={Extracting tree parameters from UAV lidar data.},
            colorlinks=false,
            urlcolor=blue,
            linkcolor=magenta,
            pdfborder={0 0 0}}
\urlstyle{same}  % don't use monospace font for urls

\setcounter{secnumdepth}{0}
% Pandoc toggle for numbering sections (defaults to be off)
\setcounter{secnumdepth}{0}


% Pandoc header



\begin{document}
\begin{frontmatter}

  \title{Extracting tree parameters from UAV lidar data.}
    \author[CIRGEO Interdepartmental Research Center of Research of Geomatics]{Francesco Pirotti\corref{1}}
   \ead{francesco.pirotti@unipd.it} 
    \author[TESAF Department]{Francesco Pirotti\corref{2}}
   \ead{francesco.pirotti@unipd.it} 
      \address[Some Institute of Technology]{Department, Street, City, State, Zip}
    \address[Another University]{Department, Street, City, State, Zip}
      \cortext[1]{Corresponding Author}
    \cortext[2]{Equal contribution}
  
  \begin{abstract}
  This is the abstract.
  
  It consists of two paragraphs.
  \end{abstract}
  
 \end{frontmatter}

\emph{Text based on elsarticle sample manuscript, see
\url{http://www.elsevier.com/author-schemas/latex-instructions\#elsarticle}}

\hypertarget{introduction}{%
\section{Introduction}\label{introduction}}

\hypertarget{installation}{%
\paragraph{Installation}\label{installation}}

Laser scanner data are becoming increasingly available and with larger
volume.

This is due to market-driven improvement of technology, making sensors
lighter and more accurate.

Terrestrial laser scanners (TLS) provide faster and more dense scans.
Productivity has also increased due to light batteries and lighter
sensors. Both factors allow more scans per day, thus surveying areas
with a denser point cloud.

For some years now laser scanners are deployed on remotely piloted
aircraft systems (RPAS), sometimes referred to as unmanned airborne
vehicles (UAVs) or simply ``drones''. Normal aircrafts require a minimum
relative flight height (RFH), usually arount 150 m above terrain. RFH is
lower for RPAS, depending on the national regulations, allowing a much
denser point cloud.

For the reasons stated above, it is not uncommon for post-processing to
deal with very dense point clouds and larger volumes of data.

Forestry has greatly benefitted from laser scanning technology, thanks
to laser's ability to go ``behind'' foliage. This is possible due to
gaps in the canopy, that allow laser signals to be only partially
reflected and partially pass through to reach other objects ``behind''
the first. This is often referred to as ``penetration'' capability of
laser scanners in canopies. When scanning vegetation cover from above,
this allows a few laser signals to be reflected by the terrain surface.
Having samples of the positions of points on the terrain surface is
fundamental to estimate a digital terrain model (DTM), that is required
for several other models. Hydrological modelling requires accurate DTM,
as well as forest modelling uses the DTM and DSM to extract the
normalized DSM (nDSM). The latter is also known as canopy height model
(CHM) when only vegetation covers the area.

Forest management fundamentally requires the spatial distribution of
trees and information on their specites, height, diameter, volume and
biomass. In the case of point clouds from laser scanning, the height can
be measured with significant accuracy, depending on the density of
points at bottom (DTM) and on the top of the canopy (DSM), in particular
the tree apex.

In the case of TLS, tree diameter can also be measured and the tree
branch structure can be extracted by segmentation {[}{]}. When laser
scanning is done from aerial supports, such as RPAS or normal aircrafts,
the height of the tree is more easily extracted.

As mentioned, new technologies allow for much denser datasets, in
particular from RPAS: The high density dataset from aerial laser
scanning becomes more similar to the common density of TLS scans. Of
course the two tecnologies provide a different distribution of point
density, due to the scan geometry.

Fast and streamlined methods for extracting information from dense point
clouds is a topic worth investigating.

\hypertarget{materials-and-method}{%
\section{Materials and method}\label{materials-and-method}}

The work uses a voxel-based density distribution to initiate tree
detection and successive segmentation and parameter definition. n octree
structure and

Here are two sample references: Feynman and Vernon Jr. (1963; Dirac,
1953).

\hypertarget{references}{%
\section*{References}\label{references}}
\addcontentsline{toc}{section}{References}

\hypertarget{refs}{}
\leavevmode\hypertarget{ref-Dirac1953888}{}%
Dirac, P., 1953. The lorentz transformation and absolute time. Physica
19, 888--896.
doi:\href{https://doi.org/10.1016/S0031-8914(53)80099-6}{10.1016/S0031-8914(53)80099-6}

\leavevmode\hypertarget{ref-Feynman1963118}{}%
Feynman, R., Vernon Jr., F., 1963. The theory of a general quantum
system interacting with a linear dissipative system. Annals of Physics
24, 118--173.
doi:\href{https://doi.org/10.1016/0003-4916(63)90068-X}{10.1016/0003-4916(63)90068-X}


\end{document}


